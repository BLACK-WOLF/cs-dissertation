%!TeX root=TermsOfReference.tex
\section{Background}
This document demonstates the structure of a proposed Terms of Reference document written in \LaTeX.

This should describe the ``context'' of the proposed project and answer the question, ``Why this project?'', both from your own perspective as a student undertaking a final year computing project, and that of any client.  It should show what makes this proposal a worthwhile computing final year project.  \emph{It must make clear both the application area or area of investigation and the computing aspects of your work}.

\LaTeX\ and its companion \BibTeX\ are a good pair of tools for writing your documentation.

Don't forget to reference background material.  \BibTeX\ makes this simple.
You can have one or more \path{.bib} files, these are  easy to populate from academic search tools, such as Google-scholar, and the Library's search facilities.  A bibliography file looks something like figure \ref{bibtex-sample}

\begin{figure}
\begin{tcolorbox}{}
\begin{minted}{bibtex}
@book{k+r,
	Author = {Brian Kerngihan and Denis Ritche},
	Edition = {Second},
	Publisher = {Prentice Hall},
	Title = {The C Programming Language},
	Year = 1988
}
\end{minted}
\end{tcolorbox}
\caption{Sample \BibTeX\ content}
\label{bibtex-sample}
\end{figure}

\section{Proposed Work}
.  The project must exhibit a level of difficulty appropriate to final year honours BSc work, and be of a size that can be attempted in the time available; this section should define the topic and project work in enough detail for the markers to be sure that it is suitable. The more detail and discussion you produce at this stage, the stronger the foundation for the actual project work.

You should emphasise the computing aspects you expect to be involved in, including those specifically relevant to your programme.  Remember that you are undertaking the project as part of a BSc programme in a computing-related discipline, and avoid being side-tracked into areas that are not relevant to your course.) 

\section{Aims and Objectives}
\subsection{Aims}
\subsection{Objectives}
\section{Skills}
\section{Resources}
\subsection{Hardware}
\subsection{Software}

\section{Structure and Contents of the Report}
\subsection{Report Structure}
\subsection{List of Appendices}

\section{Marking Scheme}


\clearpage
\section{Project Plan}~
\noindent
\rotatebox{90}{%!TeX root=TermsOfReference.tex

% A lot of the settings here are tuned to fit a landscape gantt chart into
% an A4 piece of paper.
\begin{ganttchart}[
time slot format=little-endian,
calendar week text=\currentweek,
x unit=2.4pt,
y unit chart=14pt,
y unit title=12pt,
title label font=\scriptsize,
bar top shift=.15,
bar height=0.7,
milestone label font = \small,
group label font = {\tiny\bfseries},
group inline label node/.append style=centered,
hgrid=true,vgrid={*6{draw=none},dotted},
region/.style={inline,group peaks width=2,
  group peaks height=0.25, group height=0.5,
  group top shift=0.2 ,group/.append style={fill=#1}},
milestone left shift=0,
milestone right shift=1,
ms/.style={inline,
    milestone inline label node/.append style={#1=0pt}}
]{11/9/17}{18/05/18} %<- Dates Gantt Chart runs from and to

\gantttitlecalendar{year,month,week=1}\\
% Highlight Smesters and Vactions
\ganttgroup[region=blue!10]{Sem 1}{11/9/17}{8/12/17}
\ganttgroup[region=red!50]{Christmas}{11/12/17}{1/1/18}
\ganttgroup[region=blue!50]{Sem 2}{15/1/18}{23/3/18}
\ganttgroup[region=green!25]{Easter}{26/3/18}{13/4/18}
\ganttgroup[region=blue!50]{Sem 2}{16/04/18}{27/04/18}\\

% Project Deadlines (from the slides)
\ganttmilestone[]{TOR}{13/10/17}
\ganttmilestone[ms=left]{\bfseries TOR}{10/11/17}
\ganttmilestone[ms=right]{Draft Analysis}{24/11/17}
\ganttmilestone[ms=left]{Submit}{19/4/18}
\ganttnewline[thick]

% --Tasks go here
% put in a title, a start date, end date...
\ganttbar{TOR}{11/9/17}{13/10/17}
\ganttbar[inline]{\emph{revise}}{15/10/17}{10/11/17}\\
\ganttbar{Analysis}{18/9/17}{24/11/17}\\
\ganttbar{Design}{31/10/17}{17/1/18}
\ganttmilestone[ms=right]{Build complete (Obj \ref{write-code})}{18/1/18}\\
\ganttbar{Exploration}{22/9/17}{15/11/17}
\end{ganttchart}}
