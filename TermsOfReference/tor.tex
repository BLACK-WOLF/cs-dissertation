%!TeX root=TermsOfReference.tex
\title{KV6003: Individual Project\\ Project Terms Of Reference}
\author{Cuthfbert Mutumbwa\\ Design and Implement a Rendering Engine}
\date{\today}

\maketitle

\section{Background}
Creating A rendering engine is a task that requires one to understand multiple different topics and subjects, for example not only will i need to know how lighting works, but how it interacts with other objects, and how to represent it on screen. I will also need to learn how textures work, how to wrap them around an object and how the light will interact with the texture wrapped around the object.

In college i used two piece of software that are called Unreal Engine and Blender. Unreal Engine was a game engine, and Blender was a 3D Creation suite that can be used as a rendering engine.
My first question when using these softwares where "How does one Create 3D graphics software", then i realised i also was unsure on how someone designs these software.

I want to create a rendering engine that uses core rendering techniuse and features, and Software engineering practices to design the engine. for design I have opted to use different types of litrature such as Designing a Modern Rendering engine \citep{designengine}.  I will also litrature to choose which type of technique I will use to implement my features, this is because there are multiple different ways to implemnt features such as shadows, lighting, and even the way the engine will render a scene.

Rendering engines are still or more specifically rendering is a very interesting field of study. New rendering features are being made to improve quality in 3D animated films, and recently Nvidia released a new Graphics Card to allow Real-Time raytracing in Video Games\footnote{\url{https://www.pcgamer.com/uk/what-is-ray-tracing/}}. Such Acheivments is one that is truly impressive as raytracing is known for being computationaly expensive. 


\section{Proposed Work}
I will be using C++ as my programming language of choice,and OpenGL which is a graphics API that will allow me to create graphics progra
The goal for creating a rendeirng engine is to manage graphical effects, such as lighting, shadows, texture mapping, and to apply them to a 3D scene and have them interact with other 3D objects. A rendering engine should not be confused with a game engine which would have me implement a physics system, and an animation system aswell.

I wish to build an engine with a 3D scene that will showcase various graphics effects. I want to be able to move around the 3D space using a camera, so i can be able to look at the scene at different perspective.I want to be able to turn toggle of, and on certain features to be able to show what the scene will look like without them, for example toggling off shadows to show how they may impact the scene visually.


\section{Aims and Objectives}
Aims
\begin{itemize}
  \item To Build a Rendering Engine
  \item To design a maintanable piece of software using modern design principles
\end{itemize}

Objectives
\begin{enumerate}
  \item Use software engineering design principles to implement the renderer.
  \item Learn the required theory that will be needed to further understand my project.
  \item Implement lighting in scene.
  \item Implement Shadows in scene.
  \item Implement Shaders in scene.
  \item Implement Skybox in scene.
  \item Implement Reflections in scene.
  \item Implement Texturing in scene.
  \item Implement a 3D scene enviroment to showcase the implemented features.
  \item Implement Model Loading into scene.
  \item Test and debug the system using tools specific for graphics programming.  
\end{enumerate}
\section{Skills}
\begin{enumerate}
	\item Programming in C/C++ (from module KF4006 )
	\item Software Engineering (from module KF5012)
	\item Algorithms (from module KF5008) 
	\item OpenGL API (Learnt enough to build a 3D scene) 
	\item Computer Graphics Theory (Currently Learning in Module KF6018)
\end{enumerate}

\section{Resources}
Hardware
\begin{itemize}
  \item Computer
  
\end{itemize}
Software
\begin{itemize}
  \item Visual Studio
  \item GPU debugger
\end{itemize}

\section{Structure and Contents of Project Report}
Planned Chapters
\begin{enumerate}
  \item Designing the Renderer
  \item Rendering Engine Theory
  \item Implementing the Rendering Engine
  \item Creating the Scene enviroment
  \item Testing and Debugging using bespoke Software
  \item Evaluating the Rendering Engine
\end{enumerate}

\section{Marking Scheme}
\paragraph{Project Type}Software Engineering projects

\paragraph{Project Report} 
Analysis 
\begin{enumerate}
  \item Designing the Renderer
  \item Analysing Rendering Techniques
\end{enumerate}

Synthesis 
\begin{enumerate}
  \item Implementing the Rendering Engine
  \item Creating the Scene enviroment
  \item Testing and Debugging using bespoke Software
\end{enumerate}

Evaluation
\begin{enumerate}
    \item Evaluating the Rendering Engine
\end{enumerate}


\paragraph{Product}
Deliverables list
\begin{enumerate}
    \item Class Diagram
    \item Use Case Diagram
    \item Rendering Engine Source Code 
    \item Test Plans
\end{enumerate}

\section{List of Appendices}
Will Add soon

\section{Project Type}
This will be a Software engineering project
\section{Project Plan}
\noindent
\rotatebox{90}
{\input{Gantt}}

